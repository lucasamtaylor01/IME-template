\section{Exemplos com código} % Seções são adicionadas para organizar sua apresentação em blocos discretos, todas as seções e subseções são automaticamente exibidas no índice como uma visão geral da apresentação, mas NÃO são exibidas como slides separados.

%------------------------------------------------
\begin{frame}[fragile]
    \frametitle{Python}
    
    \begin{python}
def calcular_dobro(x):
    """Retorna o dobro do número"""
    return 2 * x

# Testando a função
numero = 5
resultado = calcular_dobro(numero)
print(f"O dobro de {numero} é {resultado}")
    \end{python}
\end{frame}

%------------------------------------------------
\begin{frame}[fragile]
    \frametitle{C}
    
    \begin{clang}
#include <stdio.h>

int main() {
    int numero = 5;
    int dobro = 2 * numero;
    
    printf("O dobro de %d eh %d\n", numero, dobro);
    return 0;
}
    \end{clang}
\end{frame}

%------------------------------------------------
\begin{frame}[fragile]
    \frametitle{C++}
    
    \begin{cpp}
#include <iostream>
using namespace std;

int main() {
    int numero = 5;
    int dobro = 2 * numero;
    
    cout << "O dobro de " << numero;
    cout << " eh " << dobro << endl;
    return 0;
}
    \end{cpp}
\end{frame}

%------------------------------------------------
\begin{frame}[fragile]
    \frametitle{R}
    
    \begin{rlang}
# Função para calcular o dobro
calcular_dobro <- function(x) {
  return(2 * x)
}

# Testando a função
numero <- 5
resultado <- calcular_dobro(numero)
print(paste("O dobro de", numero, "é", resultado))
    \end{rlang}
\end{frame}

%------------------------------------------------

\begin{frame}[fragile]
    \frametitle{Java}
    
    \begin{java}
public class Exemplo {
    public static void main(String[] args) {
        int numero = 5;
        int dobro = 2 * numero;
        
        System.out.println("O dobro de " + numero +
                         " eh " + dobro);
    }
}
    \end{java}
\end{frame}