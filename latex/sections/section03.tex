\section{Caixas coloridas}

\begin{frame}{Caixas: Teorema e Definição}
	\begin{CaixaTeo}{Teorema Fundamental da Aritmética}
		Todo inteiro $n > 1$ pode ser escrito de forma única como um produto de primos, a menos da ordem dos fatores.
	\end{CaixaTeo}
	
	\begin{CaixaDef}{Número Primo}
		Um número natural $p > 1$ é primo se seus únicos divisores positivos são $1$ e $p$.
	\end{CaixaDef}
\end{frame}

\begin{frame}{Caixas: Proposição e Lema}
	\begin{CaixaProp}{Divisibilidade}
		Se $a \mid b$ e $b \mid c$, então $a \mid c$.
	\end{CaixaProp}
	
	\begin{CaixaLema}{Lema de Euclides}
		Se um primo $p$ divide o produto $ab$, então $p$ divide $a$ ou $p$ divide $b$.
	\end{CaixaLema}
\end{frame}

\begin{frame}{Caixas: Corolário e Observação}
	\begin{CaixaCoro}{Infinidade de Primos}
		Existem infinitos números primos.
	\end{CaixaCoro}
	
	\begin{CaixaObs}{Máximo Divisor Comum}
		O máximo divisor comum de dois inteiros pode ser calculado pelo Algoritmo de Euclides.
	\end{CaixaObs}
\end{frame}

\begin{frame}{Caixas: Exemplo e Nota}
	\begin{CaixaExem}{MDC com Algoritmo de Euclides}
		Para calcular $\gcd(48,18)$:
		\begin{equation*}
			48 = 2 \cdot 18 + 12,\quad 18 = 1 \cdot 12 + 6,\quad 12 = 2 \cdot 6 + 0.
		\end{equation*}
		Logo, $\gcd(48,18) = 6$.
	\end{CaixaExem}
	
	\begin{CaixaObs}{Fato curioso}
		O número $26$ é o único número natural que está entre um cubo perfeito e um quadrado perfeito.
	\end{CaixaObs}
	
\end{frame}
