%%%%%%%%%%%%%%%%%%%%%%%%%%%%%%%%%%%%%%%%%
% Beamer Presentation - LaTeX Template
% Version 2.0 (March 8, 2022)
% Original Template: https://www.LaTeXTemplates.com
% Author: Vel (vel@latextemplates.com)
% License: CC BY-NC-SA 4.0

% Este modelo de apresentação foi 
% criado a partir do modelo de Giovanni Spadaro.
% Disponível em: https://github.com/Giovo17/presentation-template-unict-lm-data
%
% Adaptado por Lucas Amaral Taylor para criar uma versão especial 
% para os alunos de Matemática e Estatística da USP (IME-USP).
% Disponível em: https://github.com/lucasamtaylor01/IME-template
%%%%%%%%%%%%%%%%%%%%%%%%%%%%%%%%%%%%%%%%%

%----------------------------------------------------------------------------------------
% CLASSE DO DOCUMENTO E CONFIGURAÇÕES BÁSICAS
%----------------------------------------------------------------------------------------
\documentclass[
    11pt,               % Tamanho padrão da fonte
    % t,                % Alinhar verticalmente ao topo
    %aspectratio=169,   % Definir proporção 16:9
]{beamer}
\graphicspath{{img/}}         % Define o diretório das imagens

%----------------------------------------------------------------------------------------
% PACOTES NECESSÁRIOS
%----------------------------------------------------------------------------------------
\usepackage{
    booktabs,     % Melhora a aparência das linhas em tabelas
    palatino,     % Define Palatino como fonte principal
    subcaption    % Suporte para subfiguras
}
\usepackage[default]{opensans}  % Define Open Sans como fonte secundária
\input{config/code_langs}       % Importa configurações para highlight de código

%----------------------------------------------------------------------------------------
% CONFIGURAÇÃO DO TEMA
%----------------------------------------------------------------------------------------
% Tema Base
\usetheme{Boadilla}                          % Define o tema principal
\useinnertheme{circles}                      % Tema interno com círculos
\useoutertheme{miniframes}                   % Tema externo com miniframes
\setbeamertemplate{navigation symbols}{}     % Remove símbolos de navegação

% Cores Personalizadas
\definecolor{primaryColor}{RGB}{20,45,105}   % Cor primária - azul escuro
\definecolor{secondaryColor}{RGB}{0,100,160} % Cor secundária - azul médio

% Configurações de Cores
\setbeamercolor{structure}{fg=primaryColor}
\setbeamercolor{palette primary}{bg=primaryColor, fg=white}
\setbeamercolor{palette secondary}{bg=secondaryColor, fg=white}
\setbeamercolor{title}{bg=primaryColor, fg=white}

% Cores do Cabeçalho e Rodapé
\setbeamercolor{headline}{bg=secondaryColor, fg=white}
\setbeamercolor{section in head/foot}{bg=primaryColor, fg=white}
\setbeamercolor{subsection in head/foot}{bg=secondaryColor, fg=white}
\setbeamercolor{author in head/foot}{bg=primaryColor, fg=white}
\setbeamercolor{title in head/foot}{bg=secondaryColor, fg=white}
\setbeamercolor{date in head/foot}{bg=primaryColor, fg=white}
\setbeamercolor{page number in head/foot}{bg=primaryColor, fg=white}

%----------------------------------------------------------------------------------------
% BIBLIOGRAFIA
%----------------------------------------------------------------------------------------
\usepackage[style=alphabetic,backend=biber]{biblatex}
\addbibresource{bibliografia.bib}

%----------------------------------------------------------------------------------------
% INFORMAÇÕES DA APRESENTAÇÃO
%----------------------------------------------------------------------------------------
\title[Título]{Título completo}          % [Versão curta]{Versão completa}
\author[Nome abreviado]{Nome}            % [Versão curta]{Nome completo}
\institute[IME-USP]{Instituto de Matemática e Estatística \\ (IME-USP)}
\date[Ano]{MÊS / ANO}

%----------------------------------------------------------------------------------------
% CAIXAS COLORIDAS (Tons equilibrados)
%----------------------------------------------------------------------------------------

\usepackage[most]{tcolorbox}
\usepackage{xcolor}

% Tons ajustados
\definecolor{Azul1}{HTML}{368FC5} % Teorema
\definecolor{Azul2}{HTML}{5EA9D6} % Definição
\definecolor{Azul3}{HTML}{8ABFE0} % Proposição
\definecolor{Azul4}{HTML}{B5D4EA} % Lema
\definecolor{Azul5}{HTML}{DBE9F4} % Corolário
\definecolor{Azul6}{HTML}{BFDCEC} % Observação
\definecolor{Azul7}{HTML}{A6CDE6} % Exemplo

% Caixa de Teorema
\newtcolorbox{CaixaTeo}[1]{colback=Azul1!10!white, colframe=Azul1, title={\textbf{Teorema:} #1}, coltitle=black}

% Caixa de Definição
\newtcolorbox{CaixaDef}[1]{colback=Azul2!10!white, colframe=Azul2, title={\textbf{Definição:} #1}, coltitle=black}

% Caixa de Proposição
\newtcolorbox{CaixaProp}[1]{colback=Azul3!10!white, colframe=Azul3, title={\textbf{Proposição:} #1}, coltitle=black}

% Caixa de Lema
\newtcolorbox{CaixaLema}[1]{colback=Azul4!10!white, colframe=Azul4, title={\textbf{Lema:} #1}, coltitle=black}

% Caixa de Corolário
\newtcolorbox{CaixaCoro}[1]{colback=Azul5!20!white, colframe=Azul5, title={\textbf{Corolário:} #1}, coltitle=black}

% Caixa de Observação
\newtcolorbox{CaixaObs}[1]{colback=Azul6!10!white, colframe=Azul6, title={\textbf{Observação:} #1}, coltitle=black}

% Caixa de Exemplo
\newtcolorbox{CaixaExem}[1]{colback=Azul7!10!white, colframe=Azul7, title={\textbf{Exemplo:} #1}, coltitle=black}

%----------------------------------------------------------------------------------------
% INÍCIO DO DOCUMENTO
%----------------------------------------------------------------------------------------
\begin{document}

% Slide de título com logo
\begin{frame}
    \begin{figure}
        \includegraphics[width=0.45\linewidth]{img/logo_IME.png}
    \end{figure}
    \titlepage
\end{frame}

% Sumário
\begin{frame}
    \frametitle{Estrutura da apresentação}
    \tableofcontents
\end{frame}

% Inclusão das seções
\section{Exemplos com texto} % Seções são adicionadas para organizar sua apresentação em blocos discretos, todas as seções e subseções são automaticamente exibidas no índice como uma visão geral da apresentação, mas NÃO são exibidas como slides separados.

%------------------------------------------------

\begin{frame}
	\frametitle{Texto corrido}
    Lorem ipsum dolor sit amet, consectetur adipiscing elit. Nullam ipsum velit, cursus quis ligula eu, malesuada aliquet massa. Quisque non convallis felis, a auctor eros. Etiam sit amet turpis a sapien pulvinar malesuada quis quis nisi. Quisque scelerisque volutpat ligula vel mollis. Nam sit amet tristique erat, sit amet cursus mi. 
\end{frame}

%------------------------------------------------

\begin{frame}
	\frametitle{Texto em tópicos numerados}
     Lorem ipsum dolor sit amet, consectetur adipiscing elit:
    \begin{enumerate}
        \item Lorem ipsum dolor sit amet.
        \item Lorem ipsum dolor sit amet.
    \end{enumerate}
	
\end{frame}

%------------------------------------------------

\begin{frame}
	\frametitle{Texto em tópicos}
     Lorem ipsum dolor sit amet, consectetur adipiscing elit:
    \begin{itemize}
        \item Lorem ipsum dolor sit amet.
        \item Lorem ipsum dolor sit amet.
    \end{itemize}
	
\end{frame}



\section{Exemplos com Equações}
\begin{frame}{Equações de Navier-Stokes}
    \footnotesize
    \begin{itemize}
        \item Forma vetorial compacta:
        \vspace{-0.2cm}
        \begin{equation*}
            \rho\left(\frac{\partial \mathbf{v}}{\partial t} + \mathbf{v} \cdot \nabla\mathbf{v}\right) = -\nabla p + \mu\nabla^2\mathbf{v} + \mathbf{f}
        \end{equation*}
        
        \item Forma expandida (3D):
        \vspace{-0.2cm}
        
        \begin{align*}
            \rho\left(\frac{\partial u}{\partial t} + u\frac{\partial u}{\partial x} + v\frac{\partial u}{\partial y} + w\frac{\partial u}{\partial z}\right) &= -\frac{\partial p}{\partial x} + \mu\left(\frac{\partial^2 u}{\partial x^2} + \frac{\partial^2 u}{\partial y^2} + \frac{\partial^2 u}{\partial z^2}\right) + f_x \\[0.3cm]
            \rho\left(\frac{\partial v}{\partial t} + u\frac{\partial v}{\partial x} + v\frac{\partial v}{\partial y} + w\frac{\partial v}{\partial z}\right) &= -\frac{\partial p}{\partial y} + \mu\left(\frac{\partial^2 v}{\partial x^2} + \frac{\partial^2 v}{\partial y^2} + \frac{\partial^2 v}{\partial z^2}\right) + f_y \\[0.3cm]
            \rho\left(\frac{\partial w}{\partial t} + u\frac{\partial w}{\partial x} + v\frac{\partial w}{\partial y} + w\frac{\partial w}{\partial z}\right) &= -\frac{\partial p}{\partial z} + \mu\left(\frac{\partial^2 w}{\partial x^2} + \frac{\partial^2 w}{\partial y^2} + \frac{\partial^2 w}{\partial z^2}\right) + f_z
        \end{align*}

        \small
        onde $\mathbf{v} = (u,v,w)$ é o campo de velocidade, $p$ é a pressão, $\rho$ é a densidade,\\
        $\mu$ é a viscosidade dinâmica e $\mathbf{f}$ representa forças externas.
    \end{itemize}
\end{frame}
\section{Exemplos com código} % Seções são adicionadas para organizar sua apresentação em blocos discretos, todas as seções e subseções são automaticamente exibidas no índice como uma visão geral da apresentação, mas NÃO são exibidas como slides separados.

%------------------------------------------------
\begin{frame}[fragile]
    \frametitle{Python}
    
    \begin{python}
def calcular_dobro(x):
    """Retorna o dobro do número"""
    return 2 * x

# Testando a função
numero = 5
resultado = calcular_dobro(numero)
print(f"O dobro de {numero} é {resultado}")
    \end{python}
\end{frame}

%------------------------------------------------
\begin{frame}[fragile]
    \frametitle{C}
    
    \begin{clang}
#include <stdio.h>

int main() {
    int numero = 5;
    int dobro = 2 * numero;
    
    printf("O dobro de %d eh %d\n", numero, dobro);
    return 0;
}
    \end{clang}
\end{frame}

%------------------------------------------------
\begin{frame}[fragile]
    \frametitle{C++}
    
    \begin{cpp}
#include <iostream>
using namespace std;

int main() {
    int numero = 5;
    int dobro = 2 * numero;
    
    cout << "O dobro de " << numero;
    cout << " eh " << dobro << endl;
    return 0;
}
    \end{cpp}
\end{frame}

%------------------------------------------------
\begin{frame}[fragile]
    \frametitle{R}
    
    \begin{rlang}
# Função para calcular o dobro
calcular_dobro <- function(x) {
  return(2 * x)
}

# Testando a função
numero <- 5
resultado <- calcular_dobro(numero)
print(paste("O dobro de", numero, "é", resultado))
    \end{rlang}
\end{frame}

%------------------------------------------------

\begin{frame}[fragile]
    \frametitle{Java}
    
    \begin{java}
public class Exemplo {
    public static void main(String[] args) {
        int numero = 5;
        int dobro = 2 * numero;
        
        System.out.println("O dobro de " + numero +
                         " eh " + dobro);
    }
}
    \end{java}
\end{frame}
\section{Conclusão} % Seções são adicionadas para organizar sua apresentação em blocos discretos, todas as seções e subseções são automaticamente exibidas no índice como uma visão geral da apresentação, mas NÃO são exibidas como slides separados.

\begin{frame}{Referências}
    \nocite{*}
    \printbibliography[heading=none]
\end{frame}
\include{sections/section04}

% Slide final
\begin{frame}
    \begin{center}
        {\Huge Fim da apresentação!}
    \end{center}
\end{frame}

\end{document}


