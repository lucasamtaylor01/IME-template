%%%%%%%%%%%%%%%%%%%%%%%%%%%%%%%%%%%%%%%%%
% Beamer Presentation
% LaTeX Template
% Version 2.0 (March 8, 2022)
%
% This template originates from:
% https://www.LaTeXTemplates.com
%
% Author:
% Vel (vel@latextemplates.com)
%
% License:
% CC BY-NC-SA 4.0 (https://creativecommons.org/licenses/by-nc-sa/4.0/)
%
%%%%%%%%%%%%%%%%%%%%%%%%%%%%%%%%%%%%%%%%%

%%%%%%%%%%%%%%%%%%%%%%%%%%%%%%%%%%%%%%%%%
% Este modelo de apresentação é uma adaptação do modelo mencionado acima. Foi criado por Giovanni Spadaro e está disponível no GitHub (https://github.com/Giovo17/presentation-template-unict-lm-data).
%
% Esta variação foi criada por Lucas Amaral Taylor com o intuito de criar uma versão especial para os alunos de Matemática e Estatística da Universidade de São Paulo. (https://github.com/lucasamtaylor01/IME-template)
%%%%%%%%%%%%%%%%%%%%%%%%%%%%%%%%%%%%%%%%%

\documentclass[
	11pt, % Define o tamanho padrão da fonte, opções incluem: 8pt, 9pt, 10pt, 11pt, 12pt, 14pt, 17pt, 20pt
	%t, % Descomente para alinhar verticalmente o conteúdo de todos os slides ao topo, em vez de centralizado
	%aspectratio=169, % Descomente para definir a proporção em 16:9, que corresponde à proporção de telas e projetores 1080p e 4K
]{beamer}

\graphicspath{{img/}} % Especifica onde procurar imagens incluídas
\usepackage[alf]{abntex2cite} % Citações no formato ABNT
\usepackage{booktabs} % Linhas de tabela aprimoradas
\usepackage{palatino} % Usa a fonte Palatino para texto serif
\usepackage[default]{opensans} % Usa a fonte Open Sans para texto sans serif

%----------------------------------------------------------------------------------------
%	SELECIONAR O TEMA DE LAYOUT E CORES
%----------------------------------------------------------------------------------------

\usetheme{Boadilla} % Tema de layout

% Definição de cores personalizadas
\definecolor{primaryColor}{RGB}{20,45,105}
\definecolor{secondaryColor}{RGB}{0,100,160}

% Aplicação das cores no tema
\setbeamercolor{structure}{fg=primaryColor}
\setbeamercolor{palette primary}{bg=primaryColor, fg=white}
\setbeamercolor{palette secondary}{bg=secondaryColor, fg=white}
\setbeamercolor{title}{bg=secondaryColor, fg=white}

\useinnertheme{circles} % Tema interno
\useoutertheme{miniframes} % Tema externo
\setbeamertemplate{navigation symbols}{} % Remove símbolos de navegação

%----------------------------------------------------------------------------------------
%	INFORMAÇÕES DA APRESENTAÇÃO
%----------------------------------------------------------------------------------------

\title[Título]{Título completo}
\author{Nome} 
\institute[IME-USP]{Instituto de Matemática e Estatística \\ (IME-USP)} 
\date[Ano]{MÊS / ANO} 

%----------------------------------------------------------------------------------------

\begin{document}

%----------------------------------------------------------------------------------------
%	SLIDE DE TÍTULO
%----------------------------------------------------------------------------------------

\begin{frame}
    \begin{figure}
        \includegraphics[width=0.45\linewidth]{img/logo_IME.png}
    \end{figure}
    \titlepage % Exibe o slide de título
\end{frame}

%----------------------------------------------------------------------------------------
%	SLIDE DE ÍNDICE
%----------------------------------------------------------------------------------------

\begin{frame}
    \frametitle{Estrutura da apresentação} % Título do slide
    \tableofcontents % Exibe o índice
\end{frame}

%----------------------------------------------------------------------------------------
%	CORPO DOS SLIDES DA APRESENTAÇÃO
%----------------------------------------------------------------------------------------

\section{Introdução} % Seções são adicionadas para organizar sua apresentação em blocos discretos, todas as seções e subseções são automaticamente exibidas no índice como uma visão geral da apresentação, mas NÃO são exibidas como slides separados.

%------------------------------------------------

\begin{frame}
	\frametitle{Introdução}
     Lorem ipsum dolor sit amet, consectetur adipiscing elit:
    \begin{enumerate}
        \item Lorem ipsum dolor sit amet.
        \item Lorem ipsum dolor sit amet.
    \end{enumerate}
	
\end{frame}

%------------------------------------------------

\section{Desenvolvimento} % Seções são adicionadas para organizar sua apresentação em blocos discretos, todas as seções e subseções são automaticamente exibidas no índice como uma visão geral da apresentação, mas NÃO são exibidas como slides separados.

\begin{frame}
	\frametitle{Desenvolvimento}
    Lorem ipsum dolor sit amet, consectetur adipiscing elit. Nam vehicula, elit at vulputate faucibus, eros justo tempus sapien, eget interdum felis justo eu leo. 

    \begin{equation*}
        1782^{12} + 1841^{12} = 1922^{12}
    \end{equation*}

    Aenean vulputate, lorem id blandit luctus, arcu lorem placerat metus, eget lobortis urna nisi eu turpis.
\end{frame}

\include{sections/Section02}

%----------------------------------------------------------------------------------------
%	SLIDE DE ENCERRAMENTO
%----------------------------------------------------------------------------------------

\begin{frame}
    \begin{center}
        {\Huge Obrigado pela atenção!}
    \end{center}
\end{frame}

%----------------------------------------------------------------------------------------

\end{document}
